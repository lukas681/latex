\documentclass{article}
\usepackage{graphicx} % Required for inserting images
% ext libs, mostly taken from previous NeurIPS submission
\usepackage{multirow}
% \usepackage{xcolor}
\usepackage{amsmath}
\let\Bbbk\relax
\usepackage{amssymb}
\usepackage{amsthm}
\DeclareMathAlphabet{\mathbbold}{U}{bbold}{m}{n}

\usepackage{mathtools}

\usepackage{hyperref}
\usepackage[capitalize,noabbrev]{cleveref} % Must be load after ams* % Must be load after ams*-packages
\usepackage{url}
\usepackage{microtype}
\usepackage{relsize}

\usepackage{algorithm}
\usepackage{algorithmicx}
\usepackage{algpseudocode}
\usepackage{arydshln}

\usepackage{listings}
\usepackage{xspace}
\usepackage{multirow}
\usepackage{thmtools}
\usepackage{thm-restate}
\usepackage{breqn}

\usepackage{booktabs}       % professional-quality tables
\usepackage{nicefrac}       % compact symbols for 1/2, etc.
\usepackage{microtype}      % microtypography
\usepackage{colortbl}       % Colorful tables
\usepackage{stackengine}
\usepackage{lipsum}
\usepackage{dsfont}
\usepackage{graphicx}
\usepackage{tabularx}
\usepackage{float}
\usepackage{mathtools}
\usepackage{mathrsfs}
\usepackage{enumitem}
\usepackage{xspace}
\usepackage[export]{adjustbox}

\usepackage{natbib}
\bibliographystyle{unsrtnat}

% Adding tikzit, Does it work with ACM?
\usepackage{conf/tikzit}
% TiKZ style file generated by TikZiT. You may edit this file manually,
% but some things (e.g. comments) may be overwritten. To be readable in
% TikZiT, the only non-comment lines must be of the form:
% \tikzstyle{NAME}=[PROPERTY LIST]

% Node styles
\tikzstyle{BLACK}=[draw=black, shape=circle, fill=black, inner sep=3pt]
\tikzstyle{DOM}=[fill={rgb,255: red,122; green,0; blue,42}, draw=black, shape=circle, inner sep=3pt]
\tikzstyle{NONE}=[fill={rgb,255: red,247; green,33; blue,212}, draw=black, shape=circle, inner sep=3pt]
\tikzstyle{NTWO}=[fill={rgb,255: red,0; green,177; blue,219}, draw=black, shape=circle, inner sep=3pt]
\tikzstyle{NTHR}=[fill={rgb,255: red,140; green,143; blue,14}, draw=black, shape=circle, inner sep=3pt]
\tikzstyle{GRAYN}=[fill={rgb,255: red,128; green,128; blue,128}, draw=black, shape=circle, inner sep=3pt]
\tikzstyle{BRGRAY}=[fill={rgb,255: red,236; green,236; blue,236}, draw=none, shape=circle, inner sep=3pt]
\tikzstyle{TEXTSTD}=[fill=white, draw=black, shape=rectangle, tikzit shape=rectangle, text width=3cm, rounded corners]
\tikzstyle{TEXTHIGH}=[fill={rgb,255: red,231; green,245; blue,255}, draw=black, shape=rectangle, tikzit shape=rectangle, text width=3cm, rounded corners]
\tikzstyle{TEXTHIGHGRAY}=[fill={rgb,255: red,220; green,220; blue,220}, draw=black, shape=rectangle, tikzit shape=rectangle, text width=3cm, rounded corners]
\tikzstyle{TEXTHIGHGREEN}=[fill={rgb,255: red,149; green,242; blue,145}, draw=black, shape=rectangle, tikzit shape=rectangle, text width=3.4cm, rounded corners]
\tikzstyle{TEXTHIGHYELLOW}=[fill={rgb,255: red,250; green,250; blue,160}, draw=black, shape=rectangle, tikzit shape=rectangle, text width=3cm, rounded corners]

% Edge styles
\tikzstyle{EDGE}=[-, fill=none]
\tikzstyle{BLUE}=[-, draw={rgb,255: red,0; green,101; blue,189}]
\tikzstyle{GRAY}=[-, fill={rgb,255: red,128; green,128; blue,128}]
\tikzstyle{DARKGREEN}=[-, fill=none, draw={rgb,255: red,19; green,100; blue,13}, line width=1.25pt]
\tikzstyle{big dash}=[-, dashed, dash pattern=on 4mm off 2mm, fill={rgb,255: red,178; green,255; blue,253}]
\tikzstyle{big dash thick}=[-, thick, dashed, dash pattern=on 4mm off 2mm, fill={rgb,255: red,178; green,255; blue,253}]
\tikzstyle{new edge style 0}=[-, fill={rgb,255: red,224; green,254; blue,255}]
\tikzstyle{FILLGREEN}=[-, fill={rgb,255: red,232; green,255; blue,207}]
\tikzstyle{FILLBLUE}=[-, fill={rgb,255: red,231; green,255; blue,255}]
\tikzstyle{FILLPURPLE}=[-, fill={rgb,255: red,255; green,233; blue,239}]
\tikzstyle{FILLDARKBLUE}=[-, fill={rgb,255: red,238; green,230; blue,255}]
\tikzstyle{FILLDARKBLUEINVISIBLE}=[-, draw=none, fill={rgb,255: red,238; green,230; blue,255}]
\tikzstyle{FILLGREENINVISBLE}=[-, draw=none, fill={rgb,255: red,232; green,255; blue,207}]
\tikzstyle{ULTRAGRAY}=[-, draw={rgb,255: red,150; green,150; blue,150}]
\tikzstyle{EDGEDASHED}=[-, dashed, fill=none]
\tikzstyle{simple directed edge}=[->, draw=black, thick, line width=0.75mm]
\tikzstyle{simple dashed directed edge}=[->, draw=black, dashed]
\tikzstyle{SLIGHTLYDASHED}=[-, dotted, fill=none, draw={rgb,255: red,128; green,128; blue,128}]
\tikzstyle{simple dashed directed edge}=[->, draw=black, line width=0.75mm, dashed]
\tikzstyle{simple dashed directed edge not thick}=[->, draw=black, line width=0.45mm, dashed]
\tikzstyle{simple}=[->, draw=black, thick]
\tikzstyle{simple}=[draw=black, fill=none, tikzit draw=black, ->]
\tikzstyle{invis}=[-, draw=none]

\usepackage[framemethod=TikZ]{mdframed}
\usepackage{footnote}
\usepackage{booktabs}
\usepackage{soul}

% Making it colorful
    \definecolor{orangeish}{HTML}{f1a340} %orange
    \definecolor{grayish}{HTML}{f7f7f7} %white-gray
    \definecolor{purpleish}{HTML}{998ec3} %purple
    \definecolor{blueish}{HTML}{004488}
    \definecolor{darkgreen}{RGB}{2,100,64} % Even darker
    % \definecolor{darkgreen}{RGB}{0,0,0,} % Even darker
    \definecolor{lightgreen}{HTML}{b2f2bb}
    \definecolor{lightblueish}{RGB}{230,244,255}
\PassOptionsToPackage{table,xcdraw,svgnames,dvipsnames}{xcolor}

\newtheorem{theorem}{Theorem}[section]
\newtheorem{lemma}[theorem]{Lemma}
\newtheorem{problem}[theorem]{Problem}
\newtheorem{notation}[theorem]{Notation}
\newtheorem{example}[theorem]{Example}
\newtheorem{definition}[theorem]{Definition}
\newtheorem{observation}[theorem]{Observation}
\newtheorem{corollary}[theorem]{Corollary}
\newtheorem{claim}[theorem]{Claim}
\newtheorem{fact}[theorem]{Fact}
\newtheorem{conj}[theorem]{Conjecture}
\newtheorem{assm}[theorem]{Assumptioo}

\title{Tight Bounds for Private Graph Data Release}
\author{Rasmus Pagh and Lukas Retschmeier}
\date{Jan 2025}
% Shortcuts that make our life easier

%\def{}{}
%\def\lra{\ensuremath\longrightarrow}
\newcommand{\algorithmnamelong}{PAMST-F(\algorithmname)\xspace}
\newcommand{\rnmBottomAlternative}{\textsc{RNM-Bottom-Alternative}\xspace}
\newcommand{\rnmBottom}{\textsc{RNM-Bottom-Standard}\xspace}

\newcommand{\rnmCompleteFast}{\textsc{Faster-Discretized-Report-Noisy-Max}\xspace}
\newcommand{\vectorize}[1]{\ensuremath{\boldsymbol{#1}}}
\newcommand{\zcdp}{z-CDP\xspace}

\newcommand{\domain}{\ensuremath{\mathbb{N}^{|\mathcal{X}|}}}
\newcommand{\R}{\ensuremath{\mathbb{R}}}
\newcommand{\N}{\ensuremath{\mathbb{N}}}
\newcommand{\G}{\ensuremath{G=(V,\E)}\xspace}
\newcommand{\Gw}{\ensuremath{G=(\mathcal{V},\mathcal{E}, w)}}
\newcommand{\err}{\ensuremath{\mathrm{Error}}}
\newcommand{\expD}{\ensuremath{{\tt Exp}}}
\newcommand{\maxExpD}{\ensuremath{\tt MaxExp}}
\newcommand{\UniD}{\ensuremath{{\tt Uni}}} % :/
\newcommand{\lapD}{\ensuremath{{\tt Lap}}}
\newcommand{\binomialD}{\ensuremath{{\tt Bin}}}
\newcommand{\normalExpD}{\ensuremath{{{\tt N}}}}
\newcommand{\priorityQueueName}{\ensuremath{{\tt PrivatePQ}}}

\newcommand{\round}[2]{{#1}\left\lfloor \dfrac{{#2}}{#1} \right\rfloor}
\newcommand{\roundInline}[2]{{#1}\left\lfloor \frac{{#2}}{#1} \right\rfloor}

\newcommand\abs[1]{\left|#1\right|}

\crefname{lstlisting}{Listing}{Listings}
\newcommand{\basicCodeStyle}{\ttfamily\small}
\newcommand{\keywordCodeStyle}{\bfseries}
\newcommand{\builtInCodeStyle}{\itshape\color{orangeish}}
\newcommand{\typesCodeStyle}{\ttfamily\small\color{blueish}}

\lstdefinelanguage{plan}{
	classoffset=0,
	morekeywords={True, False, return, null, if, in, while, do, else, case, break, continue, for, def, log, sqrt, norm},
		keywordstyle=\keywordCodeStyle,
	classoffset=1,
	morekeywords={privQuant, recenter,  scale, clip, noise},
	keywordstyle=\typesCodeStyle,
	classoffset=2,
	morekeywords={estimateStd, rankError, divideBudget, size, scope, center},
	keywordstyle=\builtInCodeStyle,
	classoffset=0,
	morecomment=[s]{/*}{*/},
	morecomment=[l]//,
	morestring=[b]",
	morestring=[b]'
}

\lstset{
	basicstyle=\basicCodeStyle,
	keywordstyle=\keywordCodeStyle,
	numberstyle=\scriptsize\sffamily,
	commentstyle={\bfseries\color{purpleish}},
	tabsize=4,
	captionpos=b,
	frame=lines,
	numbers=left,
	language=plan,
	xleftmargin=2.5em,
	framexleftmargin=2.5em,
	backgroundcolor=\color{grayish},
	breaklines=false,
	breakautoindent=false,
	postbreak=\space,
	breakindent=5pt,
	escapeinside={/*@}{@*/},
	aboveskip=3pt,
	belowskip=3pt,
	belowcaptionskip=0pt,
	morecomment=[l]{//},
	morecomment=[s]{/*}{*/},
	mathescape=true
}

% Further Typesetting
\renewcommand\vec{\mathbf}

\DeclareMathOperator*{\argmax}{\tt arg\,max}
\DeclareMathOperator*{\argmin}{\tt arg\,min}

\DeclareMathOperator*{\argmaxs}{\tt arg\,max^*}
\DeclareMathOperator*{\argmins}{\7t arg\,min^*}
\newcommand\mybox[1]{\parbox[t]{0.5\textwidth}{\raggedright$\displaystyle #1 $}}

\newcommand\Tstrut{\rule{0pt}{3.8ex}}         % = `top' strut
\newcommand\Bstrut{\rule[-0.6ex]{0pt}{0pt}}   % = `bottom' strut

\newcommand{\lukas}[1]{\textcolor{magenta}{[Lukas: #1]}}
\newcommand{\hanwen}[1]{\textcolor{orange}{[Hanwen: #1]}}
\newcommand{\rasmus}[1]{\textcolor{teal}{[Rasmus: #1]}}


% More Shortcuts
%mathcal
\newcommand{\cA}{\mathcal{A}}
\newcommand{\cB}{\mathcal{B}}
\newcommand{\cC}{\mathcal{C}}
\newcommand{\cD}{\mathcal{D}}
\newcommand{\cE}{\mathcal{E}}
\newcommand{\cF}{\mathcal{F}}
\newcommand{\cG}{\mathcal{G}}
\newcommand{\cI}{\mathcal{I}}
\newcommand{\cJ}{\mathcal{J}}
\newcommand{\cH}{\mathcal{H}}
\newcommand{\cL}{\mathcal{L}}
\newcommand{\cM}{\mathcal{M}}
\newcommand{\cO}{\mathcal{O}}
\newcommand{\cP}{\mathcal{P}}
\newcommand{\cQ}{\mathcal{Q}}
\newcommand{\cR}{\mathcal{R}}
\newcommand{\cS}{\mathcal{S}}
\newcommand{\cT}{\mathcal{T}}
\newcommand{\cU}{\mathcal{U}}
\newcommand{\cV}{\mathcal{V}}
\newcommand{\cW}{\mathcal{W}}
\newcommand{\cX}{\mathcal{X}}
\newcommand{\cY}{\mathcal{Y}}
\newcommand{\cZ}{\mathcal{Z}}
%mathscr
\newcommand{\sA}{\mathscr{A}}

% Maybe we need this in the future as well
\renewcommand{\algorithmicprocedure}{\textbf{Procedure}}
\renewcommand{\algorithmicrequire}{\quad\enspace\;\textbf{Input:}}
\renewcommand{\algorithmicensure}{\quad\enspace\;\textbf{Ensure:}}

% Requires tikz
% xshift controls the length of the bar
% yshift the vertical position
\newcommand{\doublebar}[1]{
    \begin{tikzpicture}[baseline=(X.base)]
        \node[inner sep=0] (X) {\textsf{#1}};
         \draw ([yshift=-0.35pt, xshift=0pt]X.north west) -- ([yshift=-0.35pt, xshift=-0.5pt]X.north east);
        \draw ([yshift=0.4pt, xshift=0pt]X.south west) -- ([yshift=0.4pt, xshift=-0.5pt]X.south east);
%        \draw ([yshift=-0.4pt]X.north west) -- ([yshift=-0.4pt]X.north east);
 %       \draw ([yshift=0.4pt]X.south west) -- ([yshift=0.4pt]X.south east);
    \end{tikzpicture}
}

\begin{document}
\maketitle


\section{Introduction}

Privately releasing the set of edges of a \emph{Minimum-Weight Spanning Trees}, \emph{Shortest Path Trees} or a  \emph{Minimum-Weight Matching} under edge-weight differential privacy has been introduced by \cite*{sealfon_shortest_2016}. 
In this setting, a graph topology $G = (V, E)$ is publicly known and we want to protect the weights on each individual edge.
It is natural to consider the $\ell_1$ or the $\ell_\infty$ neighboring relationship, where neighboring datasets are bounded by how much the weights can change.

While recently, there has been much focus on privately releasing an MST, tight bounds on the other problems are less understood.
While we know, that all of them can be solved by simply perturbating the input, tightness results are only known up to logarithmic factors.


Interestingly, for the MST problem, it has been shown that a simply input perturbation together with privately running \emph{any} MST algorithms yields exactly the same utility as all previous techniques. 
\subsection{Previous Work}
We will shortly survey the state of the problems individually.

\paragraph{Minimum Spanning Trees (MST).}
Tight bounds for pure DP and both the $\ell_1$ and $\ell_\infty$ neighboring relationship are known, but for approximate DP we only have tight bounds for the $\ell_\infty$ relationship.
This work solved the gap by providing tight bounds for this case as well.

\paragraph{Minimum-Weight Perfect Matchings (MWPM).}
We are unaware of any other work releasing the set of edges of a MWPM other then the original \cite{sealfon_shortest_2016}.
There has been some work on private matching like in \cite{hsu2014} or more recently in \cite{}, but those consider a different setting 
\lukas{TODO. Describe this setting and check. the STOC paper is actually before Sealfon!}

\paragraph{Shortest Path Trees (SPT).}
Again, releasing the edges of a SPT has not been subject to study after introduced by \cite{sealfon_shortest_2016} in 2016.

\subsection{Releasing Minimium Spanning tree under $\ell_1$ and $(\epsilon, \delta)$-DP}

%\input{table-of-results}
\subsection{Our Contributions}

We generalize a lower bound due to Sealfon and construct a hard, dense mst instance.

\textbf{Short Preliminaries and Notation}
That is for all neighboring vectors $\vec{X} \sim _H \vec{X'} \in [r]^{d}$ if their hamming distance is at most 1.
We write $w_{i,j}$ to denote the weight of $w_e$ for an edge $e = \{i, j\}$.
If not clear from the context, we subscript the neighborhood relationship and write $\sim_H$ for the hamming and $\sim_{\ell_1}$ for the $\ell_1$ neighborhood.

We use the operator $\in_R: 2^A  \rightarrow A$ and denote $X \in_R A$ is drawn uniformly at random from the set $A$.

\rasmus{Maybe abstract to vectors in $[r]^d$ for integers $r$ and $d$; I think we want Hamming neighborhood, i.e., differ in one entry; maybe mention the neighborhood relation in the theorem statement}

\subsection {MSTs under $\ell_1$ and approximate DP }

%\begin{theorem}\label{th:lower}
%Assuming $n>1$, if algorithm $B:[n]^{[n]} \rightarrow [n]$ is  $(\epsilon, \delta)$-dp under the hamming-distance neighboring relation, and we uniformly sample a random input $\vec{X} \in_R [n]^{n}$, then we have for all indices $i$:

%\[\Pr[B(X) \neq X_i] \geq   \dfrac{(n-1)(1-n^2\delta)}{e^\epsilon + n}\]
%\rasmus{Maybe more natural to upper bound the probability of equality?}
%\end{theorem}

We start by generalizing a proof technique that appeared in \cite{sealfon_shortest_2016} and bound the probability that any $(\epsilon, \delta)$-DP mechanism $B$ can leak some of the input bits.


\lukas{Do we rather need a $B:[n]^d \rightarrow [n]^d$?}

\lukas{Observation: Could we already use the $\ell_1$ neighborhood here and just set the neighboring $X_i +- 1$. Then we would get away with some $\ell_1$ assumption.}
\begin{theorem}
Assuming $n>1$, if algorithm $B:[n]^{d} \rightarrow [n]$ is  $(\epsilon, \delta)$-dp under the hamming neighborhood, then for $\vec{X} \in_R [n]^d$ drawn uniformly at random, we get $\Pr\left[B(\vec{X}) = \vec{X_i}\right]\leq\dfrac{e^\epsilon}{n} + \delta$.
\end{theorem}
\begin{proof}
First fix some coordinate $i\in [d]$ and  assume that $\vec{X} = (\vec{X}_{<i}, X_i, \vec{X}_{>i}) \in_{R} [n]^d$ is uniformly drawn.
Then we can bound the probability that $B(\vec{X})$ outputs some $\vec{X}_i$ by fixing entries that are not $X_i$:
\lukas{Fix -d}
\begin{align}
Pr[B(\vec{X}) = \vec{X}_i] &= 
\dfrac{1}{n^{d-1}}\sum_{\vec{X}_{<i}\in [n]^{i-1}} 
\sum_{\vec{X}_{>i} \in [n]^{d-i}}\left(\Pr_{X_i \in_R [n]}\left[B(\vec{X}_{<i},X_i,\vec{X}_{>i})= X_i\right]\right)\\
& \leq n^{1-d}\sum_{\vec{X}_{<i}\in [n]^{i-1}} \label{line:enumerate}
\sum_{\vec{X}_{>i} \in [n]^{d-i}}\left(e^\epsilon \Pr_{X_i \in_R [n]}\left[B(\vec{X}_{<i},1,\vec{X}_{>i})= X_i\right] + \delta]\right)\\
& \leq n^{1-d}\sum_{\vec{X}_{<i}\in [n]^{i-1}}  \sum_{\vec{X}_{>i} \in [n]^{d-i}}\left(e^\epsilon \dfrac{1}{n} + \delta\right)\label{line:fix}\\
&\leq \frac{e^\epsilon}{n}+ \delta
%^n\Pr[X_i=j]\Pr[B(\vec{X})_i=j|\vec{X}_i = j]\\
%&\leq \sum\limits_{j=1}^n\dfrac{1}{n}\cdot Pr[B(\vec{X}_i)_i = j] \\
%&\leq \dfrac{1}{n}\sum_{j=1}^n\left(e^\epsilon\Pr[B(\vec{X})_i = j] + \delta\right)
%\leq \dfrac{e^\epsilon}{n} + \delta\\
\end{align}
In line \ref{line:enumerate}, we use the fact that the full vector $\vec{X}$ is uniformly drawn, and we can just enumerate over all events.
%, we can simply enumerate all events while keeping $\vec{X}_i$ free.
Recall that we use the hamming neighborhood relationship.
Then finally in step \ref{line:fix}, we use the privacy guarantees of mechanism $B$ and flip to a neighboring dataset (wlog we set $X_i = 1$).

Small side note: we also get
\begin{align*}
& \Pr[B(\vec{X}) \neq j] = 1 - \Pr[B(\vec{X}) = j] \leq \dfrac{e^\epsilon}{n} + \epsilon \\
&\Leftrightarrow  \Pr[B(\vec{X}) = j] \geq \dfrac{n \cdot (1-\delta)-e^\epsilon}{n}
\end{align*}
\end{proof}

\subsection{Reduction from MST}

We are now ready to tighten the lower boind lower bound given by Sealfon in \cite{sealfon_shortest_2016} by a factor of (...).

The idea is to encode a vector $\vec{X}$ into a dense graph $G$ where privately releasing the MST exactly corresponds to privately releasing the vector.
In a second step, we transfer the previously established bound from \cref{th:lower}.
More precisely, we are going to prove the following statement.

\begin{theorem}
There exists a graph for which there exists no $(\epsilon, \delta)$-dp private mechanism $M_{V,E}:\mathcal{R}^E \rightarrow \{T|T \text{is a mst on} (V,E, \vec{W})\}$ under the $\ell_1$ neighboring relationship with error xx for sufficiently  $\delta< xx$.
\end{theorem}
\begin{proof}
\end{proof}


\paragraph{Graph Construction}
We start by describing how to encode some $\vec{X} \in [n]^d$ into a graph where privately releasing the vector corresponds exactly to privately releasing the MST of this graph.

\noindent\textit{Encoding.} We first state a function ${\cA_{\mathrm{ENCODE}}:[n]^d \rightarrow (\{V_l,V_r\}, E)}$ returning a graph $G$ with $(d + n)$ and $(n \cdot d + n - 1)$ edges E.
We construct the graph in the following way. 

\begin{enumerate}
    \item We create a path of length $n$: $V_r = \{r_1, ..., r_n\}$ be vertices encoding the possible values $[n]$ and link $r_i$ and $r_{i+1}$ with edges weighted $w_{r_i,r_{i+1}} = 0$.
    \item For each $x_i$, we create a vertex $l_i$ 
    and connect it to each $r_i$ while 
    \[
    \forall i, j \text{~we set~} w_{l_i, r_j} = \begin{cases}
       0 \text{~if~} x_i = j\\
       R \text{~else}
    \end{cases}
    \]
    \noindent We denote these $V_l = \bigcup l_i $.
\end{enumerate}

\noindent\textit{Decoding.} Let's define ${\cA_{\mathrm{DECODE}}:\{\mathrm{spanning\_trees}(G)}\} \rightarrow \{1, ..., n\}^n$ that takes an MST on this instance and converts it back to a vector in $[n]^d$.
Let 
\[
x_i = \min\limits_{j \in [n]}{w_{l_i,r_j}}
\]

For any arbitrarily chosen mst algorithm $\cA_{\mathrm{MST}}:G \rightarrow \mathrm{spanning\_trees(G)}$ and any vector $\vec{X}$, we have $\vec{X} = \cA_{\mathrm{DECODE}}\left(\cA_{\mathrm{MST}}\left(\cA_{\mathrm{ENCODE}}\right)\right)(\vec{X})$

\lukas{Maybe a proof for that? At least some explanation}

\begin{theorem}
There exists a graph for which there exists no $(\epsilon, \delta)$-dp private mechanism $M_{V,E}:\mathcal{R}^E \rightarrow \mathrm{spanning\_trees}(G)$ under the $\ell_1$ neighboring relationship with error xx for sufficiently large $\delta$.
\end{theorem}
\begin{proof}
\end{proof}

Assuming two hamming neighboring vectors, we can bound the $\ell_1$ neighboring relationship on the encoded graph:
%\begin{lemma}
%For each $\vec{X} \sim \vec{X'}$, we have:
%\[
%||\cA_{\mathrm{ENCODE}}(\vec{X}) - %\cA_{\mathrm{ENCODE}}(\vec{X'})|| \leq 2R
%\]
%\end{lemma}
%\begin{proof}
%    Trivial.
%\end{proof}
%\begin{proof}
%Observe the 
%\end{proof}

\begin{lemma}
Let $\vec{X} \sim \vec{X'}\in [n]^d$ be two neighboring vectors under the hamming neighboring relationship.
Furthermore, let $M_{V,E}$ be an $(\epsilon, \delta)$-DP MST algorithm under the $\ell_1$ neighboring relationship.
Then
\[
X \longrightarrow M_{V,E}(\cA_{\textrm{ENCODE}}(X))
\]
is $(2R\epsilon, 2Re^{2R\epsilon}\delta)$-DP
\end{lemma}
\begin{proof}
Observe that for $\vec{X}\sim \vec{X'}$, we have 
\[
||\cA_{\mathrm{ENCODE}}(\vec{X}) - \cA_{\mathrm{ENCODE}}(\vec{X'})|| \leq 2R
\].
Notice that our graph is defined by the $\ell_1$ neighboring relationship. 
Denote $z = n(d+1)-1$.
For two given $\vec{X} \sim_H \vec{X'}$ that differ in coordinate $i$ (thus $x_i \neq x'_i$ only on coordinate $i$) we can give a chain of $\ell_1$ neighboring graph weights $W^{(z)}$, such that 
\[
\vec{W}^{(0)} =\cA_{\mathrm{ENCODE}}(\vec{X}) \sim_1 \vec{W}^{(1)} \sim_1 \cdots \sim_1 \vec{W}^{(2R-1)} \sim_1  \cA_{\mathrm{ENCODE}}(\vec{X'}) = \vec{W}^{(2R)}
\]
For $i\in [z-1]$, we inductively increase (decrease) the weights for the two edges $e = \{l_i, r_{x_i}\}$ and $e' = \{l_i, r_{x'_i}\}$ that are different in the encoding.
Let $\vec{W}_0 = \cA_{\textrm{ENCODE}}(\vec{X})$, and let
\[\vec{W}^i = \begin{cases}
    \vec{W}^{(i-1)}, \textbf{~but~} w^{(i)}_{e} = w^{(i-1)}_{e}- 1 \textbf{~if~} i\leq R\\
    \\
    \vec{W}^{(i-1)}, \textbf{~but~} w^{(i)}_{e'} = w^{(i-1)}_{e'} +  1 \textbf{~if~} i > R\\
    %\left(w^{(i-1)}_1, \cdots, w^{(i-1)}_{l_i, r_i} +1, \cdots, w^{(i-1)}_{z}\right) \textbf{~if~} i > R\\
\end{cases}
\]
Because $M$ is $(\epsilon, \delta)$-DP, we know for all $Z\subseteq E$:

\begin{align*}
\Pr[M_{V,E}(\cA_{\textrm{ENCODE}}(\vec{X}) = Z] &\leq e^\epsilon \Pr[M_{V,E}(W^{(1)}) = Z] + \delta\\
& \leq e^\epsilon \left( e^\epsilon \Pr[M_{V,E}(W^{(2)}) = Z] + \delta \right) + \delta\\
& \leq \cdots \\
& \leq e^{2R\epsilon} \left(\Pr[M_{V,E}(\cA_{\mathrm{ENCODE}}(\vec{X'}))) = Z] \right) +2R \delta e^{2R\epsilon}\\
\end{align*}
The last inequality follows because we can upper bound the sum of the deltas by  $\delta \sum\limits_{i = 0}^{2R-1} e^{i\epsilon}\leq 2R\delta e^{2R}$.

Now, we are ready to derive a contradiction.
\end{proof}

\lukas{TODO: Fill this.}

\subsection{Lowering the Upper Bound of Shortest Paths}

\begin{lemma}[Tailbound on sum of laplacians \cite{dwork_algorithmic_2014}]
Let $Y_1, \cdots Y_k \sim \lapD(b_i)$ be independent variables and let $Y = \sum_i Y_i$ and $b_\max = \max_i b_i$. Let $\tau \geq \sqrt{\sum_i(b_i)^2}$, and $0 <\lambda < \frac{2\sqrt{2}\tau^2}{b_\max}$. Then
\[
\Pr[Y > \lambda] \leq \exp\left(-\dfrac{\lambda^2}{8\tau^2}\right)
\]
\end{lemma}

%\lukas{---------- Don't read from here on. In Progress! ---------}

\begin{lemma}
For any graph $G$, the number of paths with $k$ steps starting on a vertex $v$ is at most $n^k$
\end{lemma}
\begin{proof}
By a simple combinatorial argument on a complete graph $K_n$, we get:
\[
\#\mathrm{Distinct~Paths} = \dfrac{(n-1)!}{(n-k)!} \leq n^k
\]
\end{proof}

Recall that \cite{sealfon_shortest_2016}'s graph construction takes a $P_i$ and adds two edges between each pair of vertices.
\begin{lemma}[Upper bound for path graph]
The graph in the lower bound in \cite{sealfon_shortest_2016} actually has utility $\mathcal{O}(n)$.
\end{lemma}
\begin{proof}
Assume $X_1,\cdots,X_k \sim \lapD(1/\epsilon)$ and $X = \sum X_i$ for any path $P$ of length $k$
Then:
\[
\Pr[X > c n / \epsilon] \leq \exp\left(-\dfrac{c^2n^2/\epsilon^2}{8k/\epsilon^2}\right)
 = \exp\left(-\dfrac{c^2}{8}\cdot \dfrac{n^2}{k}\right).
\]

By a union bound on all $2^n$ paths:
\[
\Pr[\exists P: X_P \geq cn /\epsilon] \leq 2^n e^{-c'n} \leq e^n e^{-c'n}  =e^{n(1-c')}
\]
which is exponentially small for $c>1$.
\end{proof}

\lukas{TODO: I guess we have to reformulate these lemmas.}
\begin{lemma}
{\color{red} This is not working! Needs some more thought.}
For all graphs $G$ and $c \in (0,1)$, adding noise drawn from $\lapD(1/\epsilon)$ is $\epsilon$-dp allows to compute paths between each pair of vertices with error less than $\mathcal{O}(n \sqrt{\log n}/\epsilon)$
\end{lemma}

\begin{proof}
We observe that any graph has at most $n^n$ distinct paths. 
WLOG assumes that we have a complete graph $K_n$.
Let $c= 2\sqrt{2 c' \log n}$ and hence, for each path of length $k$, we get
\begin{align*}
\Pr[\exists P: X_P > c' n \log n /\epsilon] &\leq n^k \exp\left(-\dfrac{c' n^2 \log n}{k}\right) \\
&=  \exp\left(n\log n - \dfrac{c' n^2 \log n}{k}\right)
\end{align*}

The longest path can have up to $n-1$ edges, then the term above is exponentially small in $n$, if $n \log n - c' n \log n < 1$.
Hence, $c' > 1$ suffices, giving the desired result.
\end{proof}

Using the same technique, we get a similar result if the diameter of the graph is bounded by $\frac{n}{\sqrt{\log n}}$.

\begin{lemma}
For all graphs $G$ where the diameter is bounded by $k \leq \frac{n}{\sqrt{\log n}}$,  adding noise drawn from $\lapD(1/\epsilon)$ is $\epsilon$-dp allows to compute paths between each pair of vertices with error less than $\mathcal{O}(n/\epsilon)$ and therefore matches the known lower bound by Sealfon \cite{sealfon_shortest_2016}.
\end{lemma} 
As previously, we have:
\begin{proof}
\begin{align*}
\Pr[\exists P: X_P > c n /\epsilon] &\leq \exp\left(k \log n \right) \exp\left(-\dfrac{c^2n^2}{8k}\right) \\
&= \exp\left(k \log n - \frac{c^2 n^2}{8k}\right) 
\end{align*}
which is negative, exactly if $k^2 \log n < \dfrac{c^2 n^2}{8}$ and hence $k < \dfrac{c n}{2\sqrt{2\log n}}$. 
Setting $c = 2\sqrt{2}$ finishes the proof.
\end{proof}

\textbf{What about the gap of }

\begin{equation*}
\scalebox{0.9}{
  \tikzfig{fig/graph}
  }
\end{equation*}

\textbf{Questions / Ideas}
\begin{enumerate}
    \item Graphs with bounded diameter?
\end{enumerate}

\section{Minimum Weight Perfect Matching}

We are now turning our attention to the following problem: 
Given a publicly known graph $G = (V, E)$ with private weight vector $\vec{w} \in \R^E$, we want to release a set of edges $M \subseteq E$ where each vertex $v$ is adjacent to exactly on edge in $M$ and minimizes the weight of $w(M)$. 
We denote $A$ as the mechanism that returns an approximately minimum weight perfect matching (MWPM) under edge-weight differential privacy, where we want to protect the individual weights.
Denote $\cM$ as the set of all possible perfect matching on $G$, and then we want for all $M \in \cM(G)$ and neighboring ($\ell_1$ and $\ell_\infty$ resp.):
\[
\Pr[A(G, \vec{w}) = M] \leq  e^\epsilon \Pr[A(G, \vec{w'}) = M] + \delta \enspace.
\]

We are interested in ensuring that the weight of the released matching is not too bad (in an additive sense).
Therefore, we want to minimize the error \footnote{Note that the bound can be stated as \emph{expected error} as well} $w(A(G, \vec{w}) - w(M^*)$, the absolute difference to the real MWPM $M^*$ with probability $\gamma$.

Recently, \citet{hladik2025} have shown tight bounds for releasing an MST under pure DP.
They used a packing argument to show that the errors of $\Theta(n \log n / \epsilon)$ and $\Theta(n^2 \log n)$ are tight for the $\ell_1$ and $\ell_\infty$ neighboring relationship.
Their main argument relies on the fact that one can greedily construct a large set of weight vectors on a very dense graph that are $\Theta(n)$ in hamming distance apart.
We will now show that the same argument hold for Minimum-Weight Perfect Matching as well and prove the following theorem

\paragraph{Matchings}: We define the \emph{hamming metric} on matchings as $d_H(M_1, M_2) = |M_1 \setminus M_2| = |M_2 \setminus M_1 |$.

We will use the indicator  function to denote the weights of our hard instance:
\begin{definition}
    $\mathbbold{1}_M(e)  = \begin{cases} 0 & \text{if~} e \in M \\ 1 & \text{otherwise} \enspace.  \end{cases}$
\end{definition}
And extend this notation to $\mathbbold{1}_M(M') = \sum_{e \in M'}\mathbbold{1}_M(e)$.
Observe that ${\mathbbold{1}_M(M) = 0}$ and $||\mathbbold{1}_{M_1}(M_2) - \mathbbold{1}_{M_2}(M_1)||_1 \leq 2 d_H(M_1, M_2)$.

\lukas{TODO: Reflect $d \in \Theta(n)$. Check constants.}

\begin{theorem}[Lower Bound Pure DP]
Fix the topology of some graph $G$.
Given a set of matchings $M \subseteq \cM(G)$, let $d \in \Theta(n)$ be such that $d_H(M_1, M_2) > d$ for all $M_1, M_2 \in S$.
let $\cA$ be $\epsilon$-differentially private. with respect to $\sim_1$. 
Then there exists weights $\vec{w}\in \R^E$, such that 

\[
\Pr\left[w(M) \leq w(M^*) + \log n / (c \cdot \epsilon)-1/4\right] \leq 1/\sqrt{|S|}
\]
where $M^*$ denotes the real minimal-weight perfect matching.
\end{theorem}

We will prove that such an exponentially sized set $S$ with $|S| \in 2^{\Theta(n \log n}$ exists later in xx.

We are using the packing argument due to xxx:
\begin{lemma}
    Let $\cW \subseteq \cX$ be a collection of datasets all at distance at most $r$ from some fixed dataset $w_0 \in \cX$, and let $\{\cL_w\}_{w \in \cW}$ be a collection fo disjoint subsets of the output space $\cM$. 
    If there is a $\epsilon$-differentially private mechanism $\cA: \cX \rightarrow \cW$ such that $\Pr[\cA(w) \in \cL_w] \geq p$ for every $w \in \cW$, then
    \[
    p \leq r^{r \epsilon} / {|\cW|}
    \]
\end{lemma}


\begin{enumerate}
    \item Can we invoke the Packing Argument for pure DP to show tight (worst-case) $\Theta(n \log n)$? Can we also show $\Theta(n^2 \log n)$ for $\ell_\infty$? YES!
    \item Greedy Algorithm https://www.ee.columbia.edu/~jelena/lec21.pdf should also give a $\mathcal{\tilde{O}}(n^{3/2})$ approximation for $\ell_\infty$ / approximate DP. Is it tight?
\end{enumerate}


\newpage
\bibliography{references}

\subsection{old stuff}
\newpage
{\color{gray}

\begin{theorem}\label{th:lower}
Assuming $n>1$, if algorithm $B:[n]^{[n]} \rightarrow [n]$ is  $(\epsilon, \delta)$-dp and we uniformly sample a random input $X\leftarrow U\left([n]\right)^{n}$, then we have for all indices $i$:

\[\Pr[B(X) \neq X_i] \geq   \dfrac{(n-1)(1-n^2\delta)}{e^\epsilon + n}\]

(((Maybe more natural to upper bound the probability of equality?)))
\end{theorem}

\begin{proof}
We consider a family of algorithms $B_{i,j}$ that returns $B(Y)$ where $Y$ is equal to $X$ except that coordinate $i$ is replaced by $j$; if $B$ is differentially private with respect to $X$ then so is $B_{i,j}$.
%We shortly denote as $(X_{-i},j)$ to quickly represent a vector that has value $j\in [n/2]$ on index $i$. (((I prefer this way of saying it: 
First, note that for $B_{i, j}$: 
%(((Here we want to reason about $B_{i,j}$)))
\begin{align*}
\Pr[B(X) = X_i]  &= \dfrac{1}{n}\sum\limits_{j=1}^{n}\Pr[B_{i,j}(\vec{X}) = j] \\
FIXME. \Pr[B(X) = X_i]  &= \dfrac{1}{n}\sum\limits_{j=1}^{n}\Pr[B_{i,j}(\vec{X}) = j] \\
&\leq \dfrac{1}{n} \sum\limits_{j=1}^{n}  \dfrac{1}{n-1} \sum_{j' \neq j}\left( e^\epsilon \Pr[B_{i, j'}(\vec{X}) = j] + \delta\right) \\
&\leq \dfrac{1}{n} \sum\limits_{j=1}^{n} \left( \dfrac{1}{n-1} \sum_{j' \neq j}\left( e^\epsilon \Pr[B_{i, j'}(\vec{X}) = j]\right) + \delta \right)\\
FIXME& \leq  \dfrac{e^\epsilon}{n^2-n} \Pr[B(\vec{X}) \neq X_i] + n^2 \delta\\
%&\leq \dfrac{e^\epsilon}{n}+n^2\delta
\end{align*}

Furthermore, because $\Pr[B(\vec{X}) = X_i] = 1 - \Pr[B(\vec{X}) \neq  X_i]$, we get

\begin{align}
 1-n^2\delta&\leq \left(\dfrac{e^\epsilon}{n(n-1)} + 1\right)\Pr[B(\vec{X}\neq X_i)]\\
&\leq \dfrac{e^\epsilon + n}{(n-1)}\Pr[B(\vec{X}\neq X_i)]
\end{align}

and hence 
\[
\\Pr[B(\vec{X}\neq X_i)] \geq \dfrac{(1-n^2\delta)(n-1)}{e^\epsilon + n}
\]
\end{proof}

A second proof with a slightly different result: 

TODO rephrase with r AND rephrase 
\begin{proof}
\begin{align*}
    1\geq \Pr[B(\vec{X})\neq \vec{X}_i] &= \sum\limits_{j\in [n]} \Pr[\vec{X}_i = j]\cdot \Pr[B(\vec{X}_i)\neq j|\vec{X}_i = j]\\
    &= \dfrac{1}{n} \sum\limits_{j\in [n]}\Pr[B_{i,j}(\vec{X}) \neq  j] \\
    &=\dfrac{1}{n} \sum_{j \in [n]} \Pr[B_{i,j}(\vec{X}) \in [n]\backslash\{j\} ]\\
    &=\dfrac{1}{n} \sum_{j \in [n]} \left(\dfrac{1}{n-1}\sum_{j' \neq j} \Pr[B_{i,j}(\vec{X}) \in [n]\backslash\{j\} ]\right)\\
    &\geq \dfrac{1}{n} \sum_{j \in [n]} \left( \dfrac{e^{-\epsilon}}{n-1}\sum_{j' \neq j} (\Pr[B_{i,j'}(\vec{X}) \in [n]\backslash\{j\}] - \delta)\right) \\
    &= -e^{-\epsilon}\delta + \dfrac{1}{n} \sum_{j' \in [n]} \left( \dfrac{e^{-\epsilon}}{n-1} \sum_{j \neq j'} \Pr[B_{i,j'}(\vec{X}) \in [n]\backslash\{j\}] \right) \\
    &= e^{-\epsilon} \left(-\delta + \sum_{j \in [n]} \sum_{j' \neq j}\dfrac{1}{n} \Pr[B_{i,j'}(\vec{X}) = j'])\right) \\
    &= e^{-\epsilon} \left(-\delta + \sum_{j \in [n]} \sum_{j' \neq j} \Pr[\vec{X}_i =  j'] \cdot \Pr[B(\vec{X})=j'|\vec{X}_i = j'])\right) \\
    &e^{-\epsilon}\left(-\delta + (n-1) \cdot \sum\limits_{j' \in [n]}\Pr[B(\vec{X})=j' \wedge \vec{X}_i = j']\right) \\
    &e^{-\epsilon}\left(-\delta + (n-1) \cdot \Pr[B(\vec{X}) = \vec{X}_i]\right)
\end{align*}
Hence, we can upper bound

\[
\Pr[B(\vec{X}=\vec{X}_i)] \leq \dfrac{e^\epsilon+\delta n}{n-1}
\]

\end{proof}

Some notes for me. Or later.
\begin{enumerate}
    \item First equality: Condition a single index on all distinct outcomes.
    \item Second: $\Pr[\vec{X_i} = j] = 1/n$ because of uniformly drawing $\vec{X}$ and conditioning can be seen as ```replacing```
    \item (3) Holds because it is just the complementary event on all different $j' \neq j$
    \item (4) Because B is dp
    \item last steps: flipping sum and reverting steps
\end{enumerate}

%(1- n^2\delta)
%(eêps+n**2+n)/

%Note that for $n=1$, this exactly matches Sealfon, 

\begin{enumerate}
    \item  Why is it necessary to sum over all $j'$. Privacy should already be given for a single change.  (((I think this is just a way of getting an expression that equals the probability of not outputting $X_i$, but maybe there is a more direct way of arguing.)))
    \item I have the feeling that the bound also holds for shortest path trees because in our construction, the MST is also the shortest path tree - essentially regardless which vertex we start with.
\end{enumerate}
}


\end{document}
