% Shortcuts that make our life easier

%\def{}{}
%\def\lra{\ensuremath\longrightarrow}
\newcommand{\algorithmnamelong}{PAMST-F(\algorithmname)\xspace}
\newcommand{\rnmBottomAlternative}{\textsc{RNM-Bottom-Alternative}\xspace}
\newcommand{\rnmBottom}{\textsc{RNM-Bottom-Standard}\xspace}

\newcommand{\rnmCompleteFast}{\textsc{Faster-Discretized-Report-Noisy-Max}\xspace}
\newcommand{\vectorize}[1]{\ensuremath{\boldsymbol{#1}}}
\newcommand{\zcdp}{z-CDP\xspace}

\newcommand{\domain}{\ensuremath{\mathbb{N}^{|\mathcal{X}|}}}
\newcommand{\R}{\ensuremath{\mathbb{R}}}
\newcommand{\N}{\ensuremath{\mathbb{N}}}
\newcommand{\G}{\ensuremath{G=(V,\E)}\xspace}
\newcommand{\Gw}{\ensuremath{G=(\mathcal{V},\mathcal{E}, w)}}
\newcommand{\err}{\ensuremath{\mathrm{Error}}}
\newcommand{\expD}{\ensuremath{{\tt Exp}}}
\newcommand{\maxExpD}{\ensuremath{\tt MaxExp}}
\newcommand{\UniD}{\ensuremath{{\tt Uni}}} % :/
\newcommand{\lapD}{\ensuremath{{\tt Lap}}}
\newcommand{\binomialD}{\ensuremath{{\tt Bin}}}
\newcommand{\normalExpD}{\ensuremath{{{\tt N}}}}
\newcommand{\priorityQueueName}{\ensuremath{{\tt PrivatePQ}}}

\newcommand{\round}[2]{{#1}\left\lfloor \dfrac{{#2}}{#1} \right\rfloor}
\newcommand{\roundInline}[2]{{#1}\left\lfloor \frac{{#2}}{#1} \right\rfloor}

\newcommand\abs[1]{\left|#1\right|}

\crefname{lstlisting}{Listing}{Listings}
\newcommand{\basicCodeStyle}{\ttfamily\small}
\newcommand{\keywordCodeStyle}{\bfseries}
\newcommand{\builtInCodeStyle}{\itshape\color{orangeish}}
\newcommand{\typesCodeStyle}{\ttfamily\small\color{blueish}}

\lstdefinelanguage{plan}{
	classoffset=0,
	morekeywords={True, False, return, null, if, in, while, do, else, case, break, continue, for, def, log, sqrt, norm},
		keywordstyle=\keywordCodeStyle,
	classoffset=1,
	morekeywords={privQuant, recenter,  scale, clip, noise},
	keywordstyle=\typesCodeStyle,
	classoffset=2,
	morekeywords={estimateStd, rankError, divideBudget, size, scope, center},
	keywordstyle=\builtInCodeStyle,
	classoffset=0,
	morecomment=[s]{/*}{*/},
	morecomment=[l]//,
	morestring=[b]",
	morestring=[b]'
}

\lstset{
	basicstyle=\basicCodeStyle,
	keywordstyle=\keywordCodeStyle,
	numberstyle=\scriptsize\sffamily,
	commentstyle={\bfseries\color{purpleish}},
	tabsize=4,
	captionpos=b,
	frame=lines,
	numbers=left,
	language=plan,
	xleftmargin=2.5em,
	framexleftmargin=2.5em,
	backgroundcolor=\color{grayish},
	breaklines=false,
	breakautoindent=false,
	postbreak=\space,
	breakindent=5pt,
	escapeinside={/*@}{@*/},
	aboveskip=3pt,
	belowskip=3pt,
	belowcaptionskip=0pt,
	morecomment=[l]{//},
	morecomment=[s]{/*}{*/},
	mathescape=true
}

% Further Typesetting
\renewcommand\vec{\mathbf}

\DeclareMathOperator*{\argmax}{\tt arg\,max}
\DeclareMathOperator*{\argmin}{\tt arg\,min}

\DeclareMathOperator*{\argmaxs}{\tt arg\,max^*}
\DeclareMathOperator*{\argmins}{\7t arg\,min^*}
\newcommand\mybox[1]{\parbox[t]{0.5\textwidth}{\raggedright$\displaystyle #1 $}}

\newcommand\Tstrut{\rule{0pt}{3.8ex}}         % = `top' strut
\newcommand\Bstrut{\rule[-0.6ex]{0pt}{0pt}}   % = `bottom' strut

\newcommand{\lukas}[1]{\textcolor{magenta}{[Lukas: #1]}}
\newcommand{\hanwen}[1]{\textcolor{orange}{[Hanwen: #1]}}
\newcommand{\rasmus}[1]{\textcolor{teal}{[Rasmus: #1]}}


% More Shortcuts
%mathcal
\newcommand{\cA}{\mathcal{A}}
\newcommand{\cB}{\mathcal{B}}
\newcommand{\cC}{\mathcal{C}}
\newcommand{\cD}{\mathcal{D}}
\newcommand{\cE}{\mathcal{E}}
\newcommand{\cF}{\mathcal{F}}
\newcommand{\cG}{\mathcal{G}}
\newcommand{\cI}{\mathcal{I}}
\newcommand{\cJ}{\mathcal{J}}
\newcommand{\cH}{\mathcal{H}}
\newcommand{\cL}{\mathcal{L}}
\newcommand{\cM}{\mathcal{M}}
\newcommand{\cO}{\mathcal{O}}
\newcommand{\cP}{\mathcal{P}}
\newcommand{\cQ}{\mathcal{Q}}
\newcommand{\cR}{\mathcal{R}}
\newcommand{\cS}{\mathcal{S}}
\newcommand{\cT}{\mathcal{T}}
\newcommand{\cU}{\mathcal{U}}
\newcommand{\cV}{\mathcal{V}}
\newcommand{\cW}{\mathcal{W}}
\newcommand{\cX}{\mathcal{X}}
\newcommand{\cY}{\mathcal{Y}}
\newcommand{\cZ}{\mathcal{Z}}
%mathscr
\newcommand{\sA}{\mathscr{A}}

% Maybe we need this in the future as well
\renewcommand{\algorithmicprocedure}{\textbf{Procedure}}
\renewcommand{\algorithmicrequire}{\quad\enspace\;\textbf{Input:}}
\renewcommand{\algorithmicensure}{\quad\enspace\;\textbf{Ensure:}}

% Requires tikz
% xshift controls the length of the bar
% yshift the vertical position
\newcommand{\doublebar}[1]{
    \begin{tikzpicture}[baseline=(X.base)]
        \node[inner sep=0] (X) {\textsf{#1}};
         \draw ([yshift=-0.35pt, xshift=0pt]X.north west) -- ([yshift=-0.35pt, xshift=-0.5pt]X.north east);
        \draw ([yshift=0.4pt, xshift=0pt]X.south west) -- ([yshift=0.4pt, xshift=-0.5pt]X.south east);
%        \draw ([yshift=-0.4pt]X.north west) -- ([yshift=-0.4pt]X.north east);
 %       \draw ([yshift=0.4pt]X.south west) -- ([yshift=0.4pt]X.south east);
    \end{tikzpicture}
}
